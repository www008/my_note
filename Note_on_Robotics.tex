% !Mode::"TeX:UTF-8" % for pdfLaTeX
\documentclass[a4paper,openany,oneside,sub4section]{book}
\special{papersize=210mm,297mm}
\usepackage[left=2.0cm,right=2.0cm,top=2.5cm,bottom=2.5cm,headsep=1.5cm,footskip=1.75cm,
            marginparwidth=1.1cm,marginparsep=2mm]{geometry} % 边注,marginparwidth区域宽度,marginparsep与正文间距离
\reversemarginpar %反转边注至页面左侧

% ------ other -------------------
\usepackage{graphicx}		% 调用插图宏包
%\usepackage{subfigure}
%\pagestyle{headings}
%\usepackage{tocloft}
%\usepackage{afterpage}      %后继页通用设置宏包
\usepackage{amsfonts,amsmath,amssymb,amsthm}	% 调用公式宏包
%\usepackage{bookmark}
\usepackage{times}			% 使用 Times New Roman 字体
\usepackage{color}			% 支持彩色
%\usepackage{verbatim}
\usepackage{indentfirst}	% 设置段落缩进.如果没有,section后面正文则顶格显示.不缩进.
\linespread{1.5}			% 以將行距加大為1.5倍

%------- CTex -----------
\usepackage[UTF8, heading = true,zihao=5,]{ctex}
\ctexset{
    section = {format = \raggedright\Large\bfseries }
}

%------- hyperref -----------
\usepackage{hyperref}
%其中backref会与biblatex中的backref冲突
\hypersetup{unicode,pdfstartview=FitH,colorlinks=true,CJKbookmarks=true,linkcolor=black} 

%---------- Color ----------
\usepackage{xcolor}
\definecolor{boxOrange}{RGB}{255,138,88}
\definecolor{coverbackground}{HTML}{FFD700}

% ------- box ---------
\usepackage[most]{tcolorbox}

%-------- tip ------------
\newenvironment{boxTip}
{\begin{tcolorbox}
[enhanced jigsaw,breakable,pad at break*=1mm, colback=black!5,colframe=boxOrange]}
{\end{tcolorbox}}

\newenvironment{tip}
               {\begin{boxTip}}
               {\end{boxTip}}

%------------ Other environments ----------
\newenvironment{summary}
{\hspace{10pt}\par\small\it
 \begin{list}{}{\leftmargin=35pt\rightmargin=25pt}
 \item\ignorespaces\advance\baselineskip -1pt}
{\end{list}\vspace{-0.5cm}}

\newenvironment{remark}
{\vspace{0.5cm}\noindent\small\it
 \marginpar{\vspace{-3mm}\includegraphics[width=1.0cm]{idea.pdf}}} %利用边注实现
{\vspace{0.5cm}}

%------- footnote --------
\usepackage[perpage]{footmisc}

%--------- Bibliography ------------
%\usepackage[backend=biber]{biblatex}
\usepackage[backend=biber,style=numeric,hyperref=true,sorting=none,backref=true]{biblatex}
\addbibresource{Note_on_Robotics.bib}

%--------- Index ---------------------
\usepackage{makeidx}    % 建立索引宏包
\makeindex              %建立关键词的索引

% ======= doc ======================
\begin{document}

\title{机器人学笔记}
\author{吴浩苗}
\date{2018.6-2018.12}

\frontmatter
\thispagestyle{empty} % 当前页不显示页码
\begin{titlepage}
	%\pagecolor{coverbackground}
	\centering
	\begin{center}
		\vspace*{6cm}
		{\par \Huge {\heiti 机器人学笔记 } }
		\vspace{1cm}
		{\par \LARGE {\lishu  吴浩苗} }
		\vspace{12cm}
        {\par \large 2018.6}
	\end{center}
\end{titlepage}
%\afterpage{\nopagecolor}

\setcounter{page}{1}
\pagenumbering{Roman}		%罗马数字页码
\tableofcontents			% 目录

\mainmatter %表示文章的正文部分,在生成目录后将从第一页开始

%\setlength{\cftchapindent}{3pt}
%\setlength{\cftsecindent}{3pt}
%\setlength{\cftsubsecindent}{3pt}
\setcounter{page}{1}
\pagenumbering{arabic}
\part{数学导论}
\cleardoublepage
    \begin{tip} “数学,是上帝用来书写宇宙的语言。”——牛顿 \end{tip}

	\chapter{矢量空间}
\begin{summary}
    对单个事物的多维度描述,可构成一个维度空间,而事物就变成了空间\footnote{此空间可以是连续的,也可以是离散的。}中的一个点。\footnotemark[1]
\end{summary}

		\section{标量\&矢量}
            \subsection{点\&空间}
                \subsubsection{点}
{\par 一切控制的基础是数学描述,所以构建机器人及其控制的核心是数学建模。\footnote{数学建模并不容易。} }

\begin{remark}
  Attension: 一段注释很 \emph{重要}。可在此多些点! 
  \index{注意点}
\end{remark}


	\chapter{刚体变换}
		\section{刚体}
        刚体\footnote{现实生活中不存在绝对的刚体。}
        参考书  \cite{Latexrumen, Hartley2004}
	\chapter{速度运动学}
		\section{选择题}
\part{建模\&控制}
	\chapter{数学建模}
		\section{选择题}
	\chapter{系统控制}
		\section{选择题}
	\chapter{实践\&应用}
		\section{选择题}

\backmatter
% bibliography, glossary and index would go here.
%\nocite{*}
\printbibliography[title={参考资料}]

\printindex %打印索引

%\newpage
%\thispagestyle{empty} % 当前页不显示页码
\BgThispage
%\AddToShipoutPicture*{\BackgroundPic}
\begin{titlepage}
\centering
\begin{center}
	\vspace*{6cm}
	{\par \Huge {\heiti 机器人学笔记 } }
	\vspace{1cm}
	{\par \LARGE {\lishu  吴浩苗} }
	\vspace{12cm}
    {\par \large 2018.6}
\end{center}

\end{titlepage}
			%调入封面子源文件cover.tex
%\pagenumbering{Roman}		%罗马数字页码
%\include{abstract}			%调入摘要子源文件abstract.tex
%\pagenumbering{arabic}		%阿拉伯数字页码
%\include{contents}			%调入创建目录子源文件contents.tex
%\pagenumbering{arabic}

\end{document} 