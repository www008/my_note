% !Mode::"TeX:UTF-8"
\documentclass[UTF8]{article} % for pdfLaTeX
\usepackage{amsmath}
\usepackage{amssymb}
\usepackage{amsthm}
% \usepackage{ctex}
\usepackage{CJKutf8} 
\usepackage{hyperref}
\hypersetup{unicode} 
% \usepackage[pdftex]{graphicx}
\usepackage{graphicx}
\usepackage{subfigure}
\newtheorem{definition}{Definition}
\newtheorem{example}{Example}
\author{吴浩苗}
\title{中文书名}
\begin{document}
    \begin{CJK}{UTF8}{gkai} 
    \maketitle
    Wow! This is my FIRST \LaTeX{} Article!

    \section{1st 中文节}
     \end{CJK}

\begin{figure}[h]
  \centering
  \includegraphics[width=2cm]{face.pdf}
  \caption{tt}
  \label{fig1}
\end{figure}
    Hello World!
    \subsection{First subsection}
    I like the first subsection.
    1+1=2, $\theta \in \Re, 1+1=2$, I know 1+1=2, $I really know 1+1=2.$
    This is in text mode, $This is in math mode,$ $This\ is\ in\ math\
in\ mode.$
    Now look at it $$2-1=1$$ I DO know more than you.
    \subsection{Second subsection}
    I don't like this subsection.
    $\frac{2011}{2012}, x_1,x_2,\ldots,x_n, a^2+b^2=c^2, x_1^2+x_2^2+\
dots+x_n^2=r^{100}, \sqrt{x+1}, \sqrt[3]{x^2+1}$

$$\frac{2011}{2012}, x_1,x_2,\ldots,x_n, a^2+b^2=c^2, x_1^2+x_2^2+\
dots+x_n^2=r^{100}, \sqrt{x+1}, \sqrt[3]{x^2+1}$$
    \subsubsection{First subsubsection}
    \paragraph{1st paragraph}
    This is the first paragraph.
    \subparagraph{1st subparagraph}
        \begin{definition}
         Definition is a environment for typing definition in \LaTeX{}.
        \end{definition}
\begin{figure}[h]
  \centering
  \includegraphics[width=3cm]{draw_.pdf}
  \caption{tt}
  \label{fig2}
\end{figure}
    \subparagraph{2nd subparagraph}
    This is the second subparagraph.
    $|A|, \|A\|, \vec{a}, \overrightarrow{AB}, \tilde{x}, \widetilde{xyz
}, \mathrm{sin}, \mathbb{RCZQ}, \mathbf{ABCD}$
    \section{Second Section}

\begin{figure}[h]
  \centering
  \includegraphics[width=3cm]{circle.pdf}
  \caption{tt}
  \label{fig3}
\end{figure}

\begin{CJK}{UTF8}{gkai}
这是一个楷体中文测试,处理简体字。
\end{CJK}

\begin{CJK}{UTF8}{gbsn}
这是一个宋体中文测试,处理简体字。
\end{CJK}

\begin{CJK}{UTF8}{gbsnlp}
中文测试,处理简体字。
\end{CJK}

\begin{CJK}{UTF8}{bkai}
這是一個big5編碼的楷體中文測試,處理繁體文字。
\end{CJK}

\begin{CJK}{UTF8}{bsmi}
這是一個个big5編碼的明體中文測試,處理繁體文字。
\end{CJK}

\end{document} 