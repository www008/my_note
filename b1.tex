% !Mode::"TeX:UTF-8" % for pdfLaTeX
\documentclass[a4paper,zihao=5,openany,oneside,sub4section]{ctexbook}
\special{papersize=210mm,297mm}
\usepackage[left=2.0cm,right=2.0cm,top=2.5cm,bottom=2.5cm,headsep=1.5cm,footskip=1.75cm]{geometry}
\usepackage{amsfonts,amsmath,amssymb,amsthm}	% 调用公式宏包
%\usepackage{bookmark}
\usepackage{times}			% 使用 Times New Roman 字体
\usepackage{color}			% 支持彩色
%\usepackage{verbatim}
\usepackage{makeidx}		% 建立索引宏包
\usepackage{hyperref}
\hypersetup{pdfstartview=FitH,colorlinks=true,CJKbookmarks=true,linkcolor=black}
\usepackage{indentfirst}	% 设置段落缩进.如果没有,section后面正文则顶格显示.不缩进.
\linespread{1.5}			% 以將行距加大為1.5倍

\usepackage{afterpage}
\usepackage{xcolor}
\definecolor{coverbackground}{HTML}{FFD700}

\usepackage{titlesec}
\titleformat{\chapter}[block]
{}
{\llap{\color{gray}\chapterNumber\thechapter
\hspace{10pt}\vline}}
{10pt}
{\formatchaptertitle}

\newcommand{\formatchaptertitle}[1]{%
\parbox[t]{\dimexpr\textwidth-10pt}{\raggedright\LARGE\scshape#1}}

\newcommand{\chapterNumber}{%
\fontsize{50}{50}\usefont{U}{eur}{b}{n}}

\ctexset{
    section = {format = \raggedright\Large\bfseries }
}

%\usepackage{hyperref}
%\hypersetup{unicode}

%\usepackage{graphicx}		% 调用插图宏包
%\usepackage{subfigure}

%\pagestyle{headings}

%\usepackage{tocloft}

\begin{document}

\title{机器人学笔记}
\author{吴浩苗}
\date{2018.6-2018.12}

\frontmatter
\thispagestyle{empty} % 当前页不显示页码
\begin{titlepage}
	%\pagecolor{coverbackground}
	\centering
	\begin{center}
		\vspace*{6cm}
		{\par \Huge {\heiti 机器人学笔记 } }
		\vspace{1cm}
		{\par \LARGE {\lishu  吴浩苗} }
		\vspace{12cm}
        {\par \large 2018.6}
	\end{center}
\end{titlepage}
\afterpage{\nopagecolor}

\setcounter{page}{1}
\pagenumbering{Roman}		%罗马数字页码
\tableofcontents			% 目录

\mainmatter %表示文章的正文部分,在生成目录后将从第一页开始

%\setlength{\cftchapindent}{3pt}
%\setlength{\cftsecindent}{3pt}
%\setlength{\cftsubsecindent}{3pt}
\setcounter{page}{1}
\pagenumbering{arabic}
\part{数学导论}
\cleardoublepage
“数学,是上帝用来书写宇宙的语言。”——牛顿
	\chapter{ {\heiti 矢量空间}  }
		\section{矢量}
“数学,是上帝用来书写宇宙的语言。”——牛顿
{\par 一切控制的基础是数学描述,所以构建机器人及其控制的核心是数学建模。}
	\chapter{刚体}
		\section{选择题}
	\chapter{相对论}
		\section{选择题}
\part{建模\&控制}
	\chapter{数学建模}
		\section{选择题}
	\chapter{系统控制}
		\section{选择题}
	\chapter{实践\&应用}
		\section{选择题}

\backmatter
% bibliography, glossary and index would go here.
\printindex %打印索引

%\newpage
%\thispagestyle{empty} % 当前页不显示页码
\BgThispage
%\AddToShipoutPicture*{\BackgroundPic}
\begin{titlepage}
\centering
\begin{center}
	\vspace*{6cm}
	{\par \Huge {\heiti 机器人学笔记 } }
	\vspace{1cm}
	{\par \LARGE {\lishu  吴浩苗} }
	\vspace{12cm}
    {\par \large 2018.6}
\end{center}

\end{titlepage}
			%调入封面子源文件cover.tex
%\pagenumbering{Roman}		%罗马数字页码
%\include{abstract}			%调入摘要子源文件abstract.tex
%\pagenumbering{arabic}		%阿拉伯数字页码
%\include{contents}			%调入创建目录子源文件contents.tex
%\pagenumbering{arabic}

%\makeindex %建立关键词的索引

\end{document} 